%!TEX root = std.tex
\section*{Motivation}

Many implementations do not support the current specification of \tcode{atomic_ref}
for non lock-free types.  Also, using an \tcode{atomic_ref} of an insufficiently aligned object 
of a lock-free type can fail silently, leading to subtle and difficult to debug
errors.

There are proposals [\textit{citation needed}] to remove non lock-free \tcode{atomic_ref}
from freestanding.  However, since implementations are not required to support
lock-free atomic operations, these proposals remove the ability of using \tcode{atomic_ref}
in portable code.

The following proposal extends the \tcode{atomic_ref} specification to allow more
implementations to fully support \tcode{atomic_ref} on objects which are not  
lock-free.  This proposal preserves the existing behavior of \tcode{atomic_ref}
in implementations which can support the current specification while enabling 
additional implementations.

\section*{Proposed Wording}

\ednote{Make the following changes in [atomics.ref.generic].} \\

\indexlibrary{\idxcode{atomic_ref}}%
\indexlibrarymember{value_type}{atomic_ref}%

\begin{addedblock}
\begin{codeblock}
namespace std {
  struct atomic_ref_assume_lock_free_t {
    explicit constexpr atomic_ref_assume_lock_free_t() noexcept = default;
  };
  inline constexpr atomic_ref_assume_lock_free_t atomic_ref_assume_lock_free{};

  struct atomic_ref_prefer_user_lock_t { 
    explicit constexpr atomic_ref_prefer_user_lock_t() noexcept = default;
  };
  inline constexpr atomic_ref_prefer_user_lock_t atomic_ref_prefer_user_lock{};
}
\end{codeblock}
\end{addedblock}


\begin{codeblock}
namespace std {

  template<class T@\added{, class LockT=\unspec}@> struct atomic_ref {
  private:
    T* ptr; // \expos
    @\added{LockT* ulock; // \expos}@
  public:
    @\added{using lock_type = LockT;}@
    using value_type = T;
    static constexpr bool is_always_lock_free = @\impdefx{whether a given \tcode{atomic_ref} type's operations are always lock-free}@;
    static constexpr size_t required_@\added{lock_free}_@alignment = @\impdefx{required alignment for \tcode{atomic_ref} type's operations}@;
    @\added{static constexpr bool can_be_lock_free = required_lock_free_alignment > 0u;}@
    
    @\added{static constexpr bool never_requires_user_lock = \impdefx{};}@

    @\added{\tcode{static bool is_lock_free(const T\&) noexcept;}}@
    @\added{\tcode{static bool requires_user_lock(const T\&) noexcept;}}@

    atomic_ref& operator=(const atomic_ref&) = delete;

    explicit atomic_ref(T&) noexcept;
    @\added{\tcode{atomic_ref(T\&, atomic_ref_assume_lock_free_t) noexcept;}}@
   
    @\added{\tcode{atomic_ref(T\&, lock_type\&) noexcept;}}@
    @\added{\tcode{atomic_ref(T\&, lock_type\&, atomic_ref_prefer_user_lock_t) noexcept;}}@
    
    atomic_ref(const atomic_ref&) noexcept;

    T operator=(T) const noexcept;
    operator T() const noexcept;

    bool is_lock_free() const noexcept;
    void store(T, memory_order = memory_order_seq_cst) const noexcept;
    T load(memory_order = memory_order_seq_cst) const noexcept;
    T exchange(T, memory_order = memory_order_seq_cst) const noexcept;
    bool compare_exchange_weak(T&, T,
                               memory_order, memory_order) const noexcept;
    bool compare_exchange_strong(T&, T,
                                 memory_order, memory_order) const noexcept;
    bool compare_exchange_weak(T&, T,
                               memory_order = memory_order_seq_cst) const noexcept;
    bool compare_exchange_strong(T&, T,
                                 memory_order = memory_order_seq_cst) const noexcept;
  };

  @\added{\tcode{template<class T>}}@
  @\added{\tcode{atomic_ref(T\&) -> atomic_ref<T>;}}@

  @\added{\tcode{template<class T>}}@
  @\added{\tcode{atomic_ref(T\&, atomic_ref_assume_lock_free_t) ->}}@
    @\added{\tcode{atomic_ref<T, atomic_ref_assume_lock_free_t>;}}@

  @\added{\tcode{template<class T, class Lock>}}@
  @\added{\tcode{atomic_ref(T\&, Lock\&) -> atomic_ref<T, Lock>;}}@

  @\added{\tcode{template<class T, class Lock>}}@
  @\added{\tcode{atomic_ref(T\&, Lock\&, atomic_ref_prefer_user_lock_t) ->}}@
    @\added{\tcode{atomic_ref<T, Lock>;}}@
}
\end{codeblock}
~\\

\ednote{Make the following changes in [atomics.ref.operations].} \\

\begin{addedblock}
\pnum
The type \tcode{lock_type} can either:
\begin{itemize}
\item be equivalent to \tcode{atomic_ref_assume_lock_free_t}, or
\item the type \tcode{lock_type} meets the \oldconcept{Lockable} requirements.  
\end{itemize}
Dianostics required if \tcode{lock_type} is equivalent to \tcode{atomic_ref_assume_lock_free_t} 
and the implementation does not provide lock-free atomic
operations for objects of type \tcode{T} aligned to \\
\tcode{required_lock_free_alignment}.

\pnum
\tcode{atomic_ref} instances referencing the same value of \tcode{ptr} and \tcode{ulock} are 
called \textit{equivalent}. Concurrent access to the same value through 
equivalent \tcode{atomic_ref} instances does not create a data race\iref{intro.races}. 
\begin{note} Concurrent access to the value directly, or through a non-equivalent
\tcode{atomic_ref} instance, can introduce a data race.
\end{note}

\pnum
For all \tcode{atomic_ref} member functions excluding static methods, constructors, 
the destructor, and \\ \tcode{is_lock_free()} the following conditionals are \tcode{true}:
\begin{itemize}
\item If \tcode{ulock} points to a valid \tcode{lock_type} object which meets the 
  \oldconcept{Lockable} requirements then 
  the implementation will use \tcode{ulock} to atomically perform these methods.
  \begin{note}If \tcode{lock_type} does not meet the \oldconcept{Lockable} requirements then 
  these methods can introduce a data race. \end{note}
\item If \tcode{ulock} is equivalent to \tcode{nullptr} and
  \tcode{requires_user_lock(obj)} is \tcode{false} then
  the implementation will ensure that these methods happen atomically.
\item Otherwise, use of any of these methods can introduce a data race. 
\end{itemize}
\end{addedblock}

\begin{itemdecl}
static constexpr bool is_always_lock_free;
\end{itemdecl}

\begin{itemdescr}
\pnum
The static data member \tcode{is_always_lock_free} is \tcode{true}
if the \tcode{atomic_ref} type's operations are always lock-free,
and \tcode{false} otherwise.
\end{itemdescr}

\begin{itemdecl}
static constexpr size_t required_@\added{lock_free_}@alignment;
\end{itemdecl}

\begin{itemdescr}
\pnum
The alignment required for an object to be referenced \added{lock-free} by an atomic reference,
which is at least \tcode{alignof(T)}. \added{If the implementation does not support
lock-free operations on objects of type \tcode{T} then \tcode{required_lock_free_alignment} 
is \tcode{0}.}

\pnum
\begin{note}
Hardware could require an object
referenced by an \tcode{atomic_ref}
to have stricter alignment\iref{basic.align}
than other objects of type \tcode{T}.
Further\added{more}, whether operations on an \tcode{atomic_ref}
are lock-free could depend on the alignment of the referenced object.
For example, lock-free operations on \tcode{std::complex<double>}
could be supported only if aligned to \tcode{2*alignof(double)}.
\end{note}
\end{itemdescr}


\begin{addedblock}
\begin{itemdecl}
static constexpr bool never_requires_user_lock;
\end{itemdecl}

\begin{itemdescr}
\pnum
Is \tcode{true} if an implementation never requires the user to provide a 
lock for objects of type \tcode{T} and \tcode{false} otherwise.
\end{itemdescr}


\begin{itemdecl}
static is_lock_free(T& obj) noexcept;
\end{itemdecl}

\begin{itemdescr}
\returns Returns \tcode{true} if atomic operations on the object referenced by
\tcode{obj} can be lock-free or if the \tcode{lock_type} type is equivalent to 
\tcode{atomic_ref_assume_lock_free_t}.
\end{itemdescr}


\begin{itemdecl}
static requires_user_lock(T& obj) noexcept;
\end{itemdecl}

\begin{itemdescr}
\returns Returns \tcode{false} if \tcode{lock_type} is equivalent to
\tcode{atomic_ref_assume_lock_free_t} or does not require the user to provide
a valid reference to a \tcode{lock_type} object.
Otherwise, returns \tcode{true} if \tcode{atomic_ref} requires the user to
provide a valid reference to a \tcode{lock_type} object 
when constructing an \tcode{atomic_ref} from \tcode{obj}. 
\end{itemdescr}
\end{addedblock}


\begin{itemdecl}
explicit atomic_ref(T& obj) noexcept;
\end{itemdecl}

\begin{itemdescr}
\pnum
\removed{\requires The referenced object shall be aligned to \tcode{required_lock_free_alignment}}.

\pnum
\effects \removed{Constructs an atomic reference that references the object.}\\
\begin{addedblock}
If \tcode{requires_user_lock(obj)} is \tcode{true} calls \tcode{std::terminate()}.
Otherwise, equivalent to:
\begin{codeblock}
  ptr = std::addressof(obj);
  ulock = nullptr;
\end{codeblock}
\end{addedblock}

\pnum
\throws Nothing.
 
\end{itemdescr}


\begin{addedblock}
\begin{itemdecl}
atomic_ref(T& obj, atomic_ref_assume_lock_free_t);
\end{itemdecl}

\begin{itemdescr}
\pnum
\expects \tcode{is_lock_free(obj)} is \tcode{true}.

\pnum
\effects Equivalent to:
\begin{codeblock}
  ptr = std::addressof(obj);
  ulock = nullptr;
\end{codeblock}

\pnum
\throws Nothing.

\end{itemdescr}
\end{addedblock}


\begin{addedblock}
\begin{itemdecl}
atomic_ref(T& obj, lock_type& lk);
\end{itemdecl}

\begin{itemdescr}

\pnum
\effects Equivalent to:
\begin{codeblock}
  ptr = std::addressof(obj);
  ulock = requires_user_lock(obj) ? std::addressof(lk) : nullptr;
\end{codeblock}

\pnum
\throws Nothing.

\end{itemdescr}
\end{addedblock}


\begin{addedblock}
\begin{itemdecl}
atomic_ref(T& obj, lock_type& lk, atomic_ref_prefer_user_lock_t);
\end{itemdecl}

\begin{itemdescr}

\pnum
\effects Equivalent to:
\begin{codeblock}
  ptr = std::addressof(obj);
  ulock = std::addressof(lk);
\end{codeblock}

\pnum
\throws Nothing.
\end{itemdescr}

\end{addedblock}


\begin{itemdecl}
atomic_ref(const atomic_ref& ref) noexcept;
\end{itemdecl}

\begin{itemdescr}
\pnum
\effects \removed{Constructs an atomic reference
that references the object referenced by \tcode{ref}.}
\begin{addedblock}Equivalent to:
\begin{codeblock}
  ptr = ref.obj;
  ulock = ref.ulock;
\end{codeblock}
\end{addedblock}

\end{itemdescr}


\begin{itemdecl}
void store(T desired, memory_order order = memory_order_seq_cst) const noexcept;
\end{itemdecl}

\begin{itemdescr}
\pnum
\removed{\requires}\added{\expects} The \tcode{order} argument shall not be
\tcode{memory_order_consume},
\tcode{memory_order_acquire}, nor
\tcode{memory_order_acq_rel}.

\pnum
\effects Atomically replaces the value referenced by \tcode{*ptr}
with the value of \tcode{desired}.
Memory is affected according to the value of \tcode{order}.
\end{itemdescr}


\begin{itemdecl}
T load(memory_order order = memory_order_seq_cst) const noexcept;
\end{itemdecl}

\begin{itemdescr}
\pnum
\removed{\requires}\added{\expects} The \tcode{order} argument shall not be
\tcode{memory_order_release} nor \tcode{memory_order_acq_rel}.

\effects  Memory is affected according to the value of \tcode{order}.

\pnum
\returns Atomically returns the value referenced by \tcode{*ptr}.
\end{itemdescr}


\begin{itemdecl}
T exchange(T desired, memory_order order = memory_order_seq_cst) const noexcept;
\end{itemdecl}

\begin{itemdescr}
\pnum
\effects Atomically replaces the value referenced by \tcode{*ptr} with \tcode{desired}.
Memory is affected according to the value of \tcode{order}.
This operation is an atomic read-modify-write operation\iref{intro.multithread}.

\pnum
\returns Atomically returns the value referenced by \tcode{*ptr}
immediately before the effects.
\end{itemdescr}


\begin{itemdecl}
bool compare_exchange_weak(T& expected, T desired,
                           memory_order success, memory_order failure) const noexcept;

bool compare_exchange_strong(T& expected, T desired,
                             memory_order success, memory_order failure) const noexcept;

bool compare_exchange_weak(T& expected, T desired,
                           memory_order order = memory_order_seq_cst) const noexcept;

bool compare_exchange_strong(T& expected, T desired,
                             memory_order order = memory_order_seq_cst) const noexcept;
\end{itemdecl}

\begin{itemdescr}
\pnum
\removed{\requires}\added{\expects} The \tcode{failure} argument shall not be
\tcode{memory_order_release} nor \tcode{memory_order_acq_rel}.

\pnum
\effects
\begin{addedblock}
When only one \tcode{memory_order} argument is supplied,
the value of \tcode{success} is \tcode{order}, and
the value of \tcode{failure} is \tcode{order}
except that a value of \tcode{memory_order_acq_rel} shall be replaced by
the value \tcode{memory_order_acquire} and
a value of \tcode{memory_order_release} shall be replaced by
the value \tcode{memory_order_relaxed}.
\end{addedblock}

\begin{addedblock}
Equivalent to atomically performing the following:
\begin{codeblock}
  alignas(std::max(sizeof(T), required_lock_free_alignment)) std::byte old[sizeof(T)];
  memcpy(old, ptr, sizeof(T));
  bool result = 0 == memcmp(std::addressof(expected), old, sizeof(T));
  if (result) memcpy(ptr, std::addressof(desired), sizeof(T));
  else memcpy(std::addressof(expected), old, sizeof(T));
  return result;
\end{codeblock}

\pnum
If return value of the operation is \tcode{true},
memory is affected according to the value of \tcode{success} and
this operation is an atomic read-modify-write operation\iref{intro.races} on
the value referenced by \tcode{*ptr}.
Otherwise memory is affected according to the value of \tcode{failure} and
this operation is an atomic load operation on \tcode{*ptr}.
\end{addedblock}

\begin{removedblock}
Retrieves the value in \tcode{expected}.
It then atomically compares the value representation of
the value referenced by \tcode{*ptr} for equality
with that previously retrieved from \tcode{expected},
and if \tcode{true}, replaces the value referenced by \tcode{*ptr}
with that in \tcode{desired}.
When only one \tcode{memory_order} argument is supplied,
the value of \tcode{success} is \tcode{order}, and
the value of \tcode{failure} is \tcode{order}
except that a value of \tcode{memory_order_acq_rel} shall be replaced by
the value \tcode{memory_order_acquire} and
a value of \tcode{memory_order_release} shall be replaced by
the value \tcode{memory_order_relaxed}.
If and only if the comparison is \tcode{false} then,
after the atomic operation,
the value in \tcode{expected} is replaced by
the value read from the value referenced by \tcode{*ptr}
during the atomic comparison.
If the operation returns \tcode{true},
these operations are atomic read-modify-write operations\iref{intro.races}
on the value referenced by \tcode{*ptr}.
Otherwise, these operations are atomic load operations on that memory.
\end{removedblock}

\begin{removedblock}
\pnum
\returns The result of the comparison.
\end{removedblock}

\pnum
\remarks A weak compare-and-exchange operation may fail spuriously.
That is, even when the contents of memory referred to
by \tcode{expected} and \tcode{ptr} are equal,
it may return \tcode{false} and
store back to \tcode{expected} the same memory contents
that were originally there.
\begin{note}
This spurious failure enables implementation of compare-and-exchange
on a broader class of machines, e.g., load-locked store-conditional machines.
A consequence of spurious failure is
that nearly all uses of weak compare-and-exchange will be in a loop.
When a compare-and-exchange is in a loop,
the weak version will yield better performance on some platforms.
When a weak compare-and-exchange would require a loop and
a strong one would not, the strong one is preferable.
\end{note}
\end{itemdescr}

~\\
~\\

\ednote{Make the following changes in [atomics.ref.int].} \\

\begin{codeblock}
namespace std {
  template<@\added{class LockT}@> struct atomic_ref<@\placeholder{integral}\added{, LockT}@> {
  private:
    @\placeholder{integral}@* ptr; // \expos
    @\added{LockT* ulock; // \expos}@
  public:
    @\added{using lock_type = LockT;}@
    using value_type = @\placeholder{integral}@;
    using difference_type = value_type;
    static constexpr bool is_always_lock_free = @\impdefx{whether a given \tcode{atomic_ref} type's operations are always lock-free}@;
    static constexpr size_t required_@\added{lock_free}_@alignment = @\impdefx{required alignment for \tcode{atomic_ref} type's operations}@;
    @\added{static constexpr bool can_be_lock_free = required_lock_free_alignment > 0u;}@
    
    @\added{static constexpr bool never_requires_user_lock = \impdefx{};}@

    @\added{\tcode{static bool is_lock_free(const T\&) noexcept;}}@
    @\added{\tcode{static bool requires_user_lock(const T\&) noexcept;}}@

    atomic_ref& operator=(const atomic_ref&) = delete;

    explicit atomic_ref(T&) noexcept;
    @\added{atomic_ref(\placeholder{integral}\&, atomic_ref_assume_lock_free_t) noexcept;}@
    
    @\added{atomic_ref(\placeholder{integral}\&, lock_type\&) noexcept;}@
    @\added{atomic_ref(\placeholder{integral}\&, lock_type\&, atomic_ref_prefer_user_lock_t) noexcept;}@
    
    atomic_ref(const atomic_ref&) noexcept;

    @\placeholdernc{integral}@ operator=(@\placeholder{integral}@) const noexcept;
    operator @\placeholdernc{integral}@() const noexcept;

    bool is_lock_free() const noexcept;
    void store(@\placeholdernc{integral}@, memory_order = memory_order_seq_cst) const noexcept;
    @\placeholdernc{integral}@ load(memory_order = memory_order_seq_cst) const noexcept;
    @\placeholdernc{integral}@ exchange(@\placeholdernc{integral}@,
                      memory_order = memory_order_seq_cst) const noexcept;
    bool compare_exchange_weak(@\placeholder{integral}@&, @\placeholder{integral}@,
                               memory_order, memory_order) const noexcept;
    bool compare_exchange_strong(@\placeholder{integral}@&, @\placeholder{integral}@,
                                 memory_order, memory_order) const noexcept;
    bool compare_exchange_weak(@\placeholder{integral}@&, @\placeholder{integral}@,
                               memory_order = memory_order_seq_cst) const noexcept;
    bool compare_exchange_strong(@\placeholder{integral}@&, @\placeholder{integral}@,
                                 memory_order = memory_order_seq_cst) const noexcept;

    @\placeholdernc{integral}@ fetch_add(@\placeholdernc{integral}@,
                       memory_order = memory_order_seq_cst) const noexcept;
    @\placeholdernc{integral}@ fetch_sub(@\placeholdernc{integral}@,
                       memory_order = memory_order_seq_cst) const noexcept;
    @\placeholdernc{integral}@ fetch_and(@\placeholdernc{integral}@,
                       memory_order = memory_order_seq_cst) const noexcept;
    @\placeholdernc{integral}@ fetch_or(@\placeholdernc{integral}@,
                      memory_order = memory_order_seq_cst) const noexcept;
    @\placeholdernc{integral}@ fetch_xor(@\placeholdernc{integral}@,
                       memory_order = memory_order_seq_cst) const noexcept;

    @\placeholdernc{integral}@ operator++(int) const noexcept;
    @\placeholdernc{integral}@ operator--(int) const noexcept;
    @\placeholdernc{integral}@ operator++() const noexcept;
    @\placeholdernc{integral}@ operator--() const noexcept;
    @\placeholdernc{integral}@ operator+=(@\placeholdernc{integral}@) const noexcept;
    @\placeholdernc{integral}@ operator-=(@\placeholdernc{integral}@) const noexcept;
    @\placeholdernc{integral}@ operator&=(@\placeholdernc{integral}@) const noexcept;
    @\placeholdernc{integral}@ operator|=(@\placeholdernc{integral}@) const noexcept;
    @\placeholdernc{integral}@ operator^=(@\placeholdernc{integral}@) const noexcept;
  };
}
\end{codeblock}

\ednote{Make the following changes in [atomics.ref.float].} \\

\begin{codeblock}
namespace std {
  template<@\added{class LockT}@> struct atomic_ref<@\placeholder{floating-point}\added{, LockT}@> {
  private:
    @\placeholder{floating-point}@* ptr; // \expos
    @\added{LockT* ulock; // \expos}@
  public:
    @\added{using lock_type = LockT;}@
    using value_type = @\placeholder{floating-point}@;
    using difference_type = value_type;
    static constexpr bool is_always_lock_free = @\impdefx{whether a given \tcode{atomic_ref} type's operations are always lock-free}@;
    static constexpr size_t required_@\added{lock_free}_@alignment = @\impdefx{required alignment for \tcode{atomic_ref} type's operations}@;
    @\added{static constexpr bool can_be_lock_free = required_lock_free_alignment > 0u;}@
    
    @\added{static constexpr bool never_requires_user_lock = \impdefx{};}@

    @\added{\tcode{static bool is_lock_free(const T\&) noexcept;}}@
    @\added{\tcode{static bool requires_user_lock(const T\&) noexcept;}}@

    atomic_ref& operator=(const atomic_ref&) = delete;

    explicit atomic_ref(T&) noexcept;
    @\added{atomic_ref(\placeholder{floating-point}\&, atomic_ref_assume_lock_free_t) noexcept;}@
    
    @\added{atomic_ref(\placeholder{floating-point}\&, lock_type\&) noexcept;}@
    @\added{atomic_ref(\placeholder{floating-point}\&, lock_type\&, atomic_ref_prefer_user_lock_t) noexcept;}@
    
    atomic_ref(const atomic_ref&) noexcept;

    @\placeholder{floating-point}@ operator=(@\placeholder{floating-point}@) noexcept;
    operator @\placeholdernc{floating-point}@() const noexcept;

    bool is_lock_free() const noexcept;
    void store(@\placeholder{floating-point}@, memory_order = memory_order_seq_cst) const noexcept;
    @\placeholder{floating-point}@ load(memory_order = memory_order_seq_cst) const noexcept;
    @\placeholder{floating-point}@ exchange(@\placeholder{floating-point}@,
                            memory_order = memory_order_seq_cst) const noexcept;
    bool compare_exchange_weak(@\placeholder{floating-point}@&, @\placeholder{floating-point}@,
                               memory_order, memory_order) const noexcept;
    bool compare_exchange_strong(@\placeholder{floating-point}@&, @\placeholder{floating-point}@,
                                 memory_order, memory_order) const noexcept;
    bool compare_exchange_weak(@\placeholder{floating-point}@&, @\placeholder{floating-point}@,
                               memory_order = memory_order_seq_cst) const noexcept;
    bool compare_exchange_strong(@\placeholder{floating-point}@&, @\placeholder{floating-point}@,
                                 memory_order = memory_order_seq_cst) const noexcept;

    @\placeholder{floating-point}@ fetch_add(@\placeholder{floating-point}@,
                             memory_order = memory_order_seq_cst) const noexcept;
    @\placeholder{floating-point}@ fetch_sub(@\placeholder{floating-point}@,
                             memory_order = memory_order_seq_cst) const noexcept;

    @\placeholder{floating-point}@ operator+=(@\placeholder{floating-point}@) const noexcept;
    @\placeholder{floating-point}@ operator-=(@\placeholder{floating-point}@) const noexcept;
  };
}
\end{codeblock}

\ednote{Make the following changes in [atomics.ref.pointer].} \\

\begin{codeblock}
namespace std {
  template<class T@\added{,class LockT}@> struct atomic_ref<T*@\added{, LockT}@> {
  private:
    T** ptr; // \expos
    @\added{LockT* ulock; // \expos}@
  public:
    @\added{using lock_type = LockT;}@
    using value_type = T*;
    using difference_type = ptrdiff_t;
    static constexpr bool is_always_lock_free = @\impdefx{whether a given \tcode{atomic_ref} type's operations are always lock-free}@;
    static constexpr size_t required_@\added{lock_free}_@alignment = @\impdefx{required alignment for \tcode{atomic_ref} type's operations}@;
    @\added{static constexpr bool can_be_lock_free = required_lock_free_alignment > 0u;}@
    
    @\added{static constexpr bool never_requires_user_lock = \impdefx{};}@

    @\added{\tcode{static bool is_lock_free(const T\&) noexcept;}}@
    @\added{\tcode{static bool requires_user_lock(const T\&) noexcept;}}@

    atomic_ref& operator=(const atomic_ref&) = delete;

    explicit atomic_ref(T&) noexcept;
    @\added{atomic_ref(T*, atomic_ref_assume_lock_free_t) noexcept;}@
   
    @\added{atomic_ref(T*, lock_type\&) noexcept;}@
    @\added{atomic_ref(T*, lock_type\&, atomic_ref_prefer_user_lock_t) noexcept;}@
    
    atomic_ref(const atomic_ref&) noexcept;

    T* operator=(T*) const noexcept;
    operator T*() const noexcept;

    bool is_lock_free() const noexcept;
    void store(T*, memory_order = memory_order_seq_cst) const noexcept;
    T* load(memory_order = memory_order_seq_cst) const noexcept;
    T* exchange(T*, memory_order = memory_order_seq_cst) const noexcept;
    bool compare_exchange_weak(T*&, T*,
                               memory_order, memory_order) const noexcept;
    bool compare_exchange_strong(T*&, T*,
                                 memory_order, memory_order) const noexcept;
    bool compare_exchange_weak(T*&, T*,
                               memory_order = memory_order_seq_cst) const noexcept;
    bool compare_exchange_strong(T*&, T*,
                                 memory_order = memory_order_seq_cst) const noexcept;

    T* fetch_add(difference_type, memory_order = memory_order_seq_cst) const noexcept;
    T* fetch_sub(difference_type, memory_order = memory_order_seq_cst) const noexcept;

    T* operator++(int) const noexcept;
    T* operator--(int) const noexcept;
    T* operator++() const noexcept;
    T* operator--() const noexcept;
    T* operator+=(difference_type) const noexcept;
    T* operator-=(difference_type) const noexcept;
  };
}
\end{codeblock}

